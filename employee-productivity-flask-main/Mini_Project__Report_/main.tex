\documentclass[12pt,oneside]{report}
\usepackage[a4paper,top=25mm,left=25mm,right=25mm,bottom=25mm]{geometry}
\usepackage{setspace}
\usepackage{graphicx}
\usepackage{float}
\usepackage{hyperref}
\usepackage{amsmath}
\usepackage{titlesec}

\onehalfspacing

\begin{document}

\begin{titlepage}
\centering

% ---------- TOP TEXT ----------
A\\[4pt]
\textbf{Mini Project Report}\\[6pt]
On\\[12pt]

{\LARGE \textbf{Employee Productivity Predictor Using Machine Learning}}\\[20pt]

Submitted in partial fulfilment for the requirement for the award of the degree\\[8pt]
Of\\[10pt]

\textbf{Bachelor of Technology}\\[6pt]
In\\[6pt]
\textbf{Computer Science and Engineering}\\[22pt]

Submitted by\\[6pt]
\textbf{Utkarsh Singh}\\
(2202840100240)\\[25pt]

% ---------- LOGO ----------
\begin{center}
    \includegraphics[width=0.40\textwidth]{logo.png}
\end{center}

% ---------- INSTITUTE DETAILS ----------
Department of Computer Science \& Engineering\\[2pt]
\textbf{UNITED INSTITUTE OF TECHNOLOGY (284)}\\[2pt]
(Affiliated to Dr. A.P.J. Abdul Kalam Technical University, Lucknow)\\[12pt]

\textbf{Session-2025-2026}

\end{titlepage}

% =========================================
% VISION & MISSION PAGE
% =========================================

\chapter*{}
\begin{center}
\vspace{-4cm}
    {\LARGE \textbf{Vision of the Department}}
\end{center}

\justifying
To be a center of excellence in the field of Computer Science and Engineering
for producing talented engineers to ethically serve constantly changing needs of
society and industry throughout their career and life.

\vspace{2.5cm}

\begin{center}
    {\LARGE \textbf{Mission of the Department}}
\end{center}

\vspace{0.5cm}

\justifying

\noindent \textbf{M1.} Accomplish excellence with committed faculty by providing theoretical 
foundation and practical skills for solving complex engineering problems in the 
state-of-the-art trends in Computer science and allied disciplines.\\[10pt]

\noindent \textbf{M2.} To foster skills and competency, generating novel ideas, entrepreneurship 
and model creations focused towards deep knowledge, interpersonal skills and 
leadership.\\[10pt]

\noindent \textbf{M3.} To develop habitude of research among faculty and students in the area of 
Computer Science \& Allied disciplines by providing the desired environment 
for addressing the needs of industry and society.\\[10pt]

\noindent \textbf{M4.} To mould the students with ethical principles in thoughts, expression and deeds.

\newpage
\chapter*{\centering \vspace{-3cm}{\textbf{Index}}}

\noindent
\large{\textbf{S.No} \hspace{3cm} \textbf{Topic} \hspace{9.5cm} \textbf{Page No.}} \\[10pt]

\textbf{1} \hspace{2.2cm} Introduction \dotfill\hspace{0.5cm} 3 \\[8pt]

\textbf{2} \hspace{2.2cm} Requirement \dotfill\hspace{0.5cm} 4 \\[8pt]

\textbf{3} \hspace{2.2cm} Conceptual Background \dotfill\hspace{0.5cm} 5--7 \\[8pt]

\textbf{4} \hspace{2.2cm} Implementation \dotfill\hspace{0.5cm} 8--12 \\[8pt]

% ----- SUBHEADINGS INDENTED -----
\hspace{1.4cm} \textbf{4.1} \hspace{1cm} Coding \dotfill \\[8pt]

\hspace{1.4cm} \textbf{4.2} \hspace{1cm} Flow Chart \dotfill \\[8pt]

% ---------------------------------

\textbf{5} \hspace{2.2cm} User Interface \dotfill\hspace{0.5cm} 13--14 \\[8pt]

\textbf{6} \hspace{2.2cm} Future Scope \dotfill\hspace{0.5cm} 15--16 \\[10pt]

\hspace{2.2cm} References \dotfill\hspace{0.5cm} 17 \\[8pt]

\hspace{2.2cm} Appendix \dotfill\hspace{0.5cm} 18 \\[8pt]

\newpage



% =========================================
% CHAPTER 1 : INTRODUCTION
% =========================================
\chapter*{\centering \vspace{-4cm} \textbf{Chapter - 1}\\[4pt] \textbf{Introduction}}
\addcontentsline{toc}{chapter}{Introduction}

\justifying

Employee productivity plays a crucial role in the growth, planning, and operational success 
of any organization. Traditionally, organizations rely on manual evaluation, supervisor 
feedback and historical performance records to estimate employee productivity. However, 
such approaches are time-consuming, less accurate, and often biased due to human judgement.

With the advancement of data-driven technologies, organizations are now adopting predictive 
analytics and machine learning models to measure employee performance more precisely. 
Machine learning enables organizations to analyse historical employee data, identify patterns 
affecting productivity and generate predictive insights which can assist in decision making.

This project focuses on developing a machine learning-based system that automatically 
predicts employee productivity using historical workplace data such as working hours, 
department, attendance, and other relevant numerical parameters. The system allows 
comparison of multiple machine learning models and helps organizations select the most 
suitable algorithm for productivity prediction based on evaluation metrics.

\textbf{Problem Statement :} Manual productivity evaluation is slow, subjective and lacks predictive 
capability. There is a requirement for a machine learning-based solution that can 
automatically analyse employee data and predict productivity accurately.

\clearpage

% =========================================
% CHAPTER 2 : REQUIREMENTS
% =========================================

\chapter*{\centering \vspace{-4cm} \textbf{Chapter - 2}\\[4pt] \textbf{Requirements}}
\addcontentsline{toc}{chapter}{Requirements}

\vspace{0.8cm}

% ---------- SOFTWARE ----------
\noindent \textbf{Software :}

\vspace{0.4cm}

\noindent 1. Programming Language Python 3.x\\[4pt]
\noindent 2. Backend Framework Flask (used for serving ML model and handling routes)\\[4pt]
\noindent 3. Machine Learning Libraries — scikit-learn, XGBoost, Pandas, NumPy\\[4pt]
\noindent 4. Data Visualization — Matplotlib, Seaborn\\[4pt]
\noindent 5. Frontend — HTML, CSS, Bootstrap\\[4pt]
\noindent 6. Deployment — Localhost (Flask server)\\[4pt]
\noindent 7. Miscellaneous — joblib (model saving/loading), CSV handling\\

\vspace{1cm}

% ---------- HARDWARE ----------
\noindent \textbf{Hardware :}

\vspace{0.4cm}

\noindent \textbf{1. Standard PC/Laptop:} With sufficient processing power and memory to support the development environment.\\[6pt]

\noindent \textbf{2. Local Database Server:} Can be installed on the development machine for testing and development purposes.\\[6pt]

\noindent \textbf{3. Display:} A standard display for UI design and testing.


\newpage



% =========================================
% CHAPTER 3 : CONCEPTUAL BACKGROUND
% =========================================

\chapter*{\centering \vspace{-4cm} \textbf{Chapter - 3}\\[4pt] \textbf{Conceptual Background}}
\addcontentsline{toc}{chapter}{Conceptual Background}

\justifying
\vspace{-0.25cm}

The concept of employee productivity prediction is rooted in the increasing need for 
organizations to measure and analyze employee performance in a more scientific and 
data-driven manner. With the growth of industries and digital technologies, manual 
evaluation of employees has become inefficient for large-scale workforce environments. 
Machine learning and data analytics provide organizations with accurate techniques to 
process employee-related information and predict productivity based on past patterns and 
performance indicators.

\vspace{0.15cm}

\noindent \textbf{Understanding Employee Productivity :} Employee productivity refers to the effectiveness 
and efficiency with which employees complete assigned tasks. Productivity depends on 
several factors such as working environment, job role, skill level, working hours, and 
inter-departmental workload. Traditionally, productivity assessment was based on supervisor 
feedback and manual judgment, which often lacked accuracy.

\vspace{0.15cm}

\noindent \textbf{Need for Analytics Based Evaluation :} Organizations today collect large volumes of 
employee-related data in digital form. Instead of relying on subjective evaluation, this data 
can be analyzed using machine learning algorithms which automatically learn productivity 
trends and predict future performance. Some important benefits include : faster 
decision-making, removal of human bias, identification of performance-related issues, 
automated productivity estimation. Thus, data analytics supports intelligent workforce 
planning and improves organizational efficiency.

\vspace{0.15cm}

\noindent \textbf{Machine Learning for Prediction :} Machine learning (ML) is a field of Artificial 
Intelligence where computers learn patterns from existing information and apply the learned 
knowledge to predict future outcomes. ML does not require explicit programming; instead, it 
uses historical data to discover hidden relationships between different variables.

\vspace{0.15cm}

\noindent \textbf{In employee productivity systems:}

\begin{itemize}
    \item  ML models learn relationships between features (input) and productivity values (output).
    \item Models continue to improve whenever more training data becomes available.
\end{itemize}

\noindent \textbf{Supervised Learning :} The prediction of employee productivity mainly uses supervised 
learning techniques, because the historical dataset already contains the actual productivity 
values. Regression models are trained using employee features and the model learns to 
estimate future values. Main supervised models applied: Linear Regression, Random Forest, 
XGBoost Regression. These models are suitable for continuous numerical prediction.

\vspace{0.15cm}

\noindent \textbf{Regression and Ensemble Models :} Regression models are commonly used to capture 
relationships between input features and productivity output values. Ensemble models like 
Random Forest and XGBoost combine multiple learners to increase prediction accuracy. 
Ensemble techniques reduce model errors, handle complex relationships and provide better 
results than single linear models. Therefore, Regression models give baseline prediction, 
Ensemble models improve performance through boosting or bagging mechanisms.

\vspace{0.15cm}

\noindent \textbf{Performance Evaluation :} Once a model is trained, model performance is tested by 
comparing predicted output with actual results. Metrics such as MAE (Mean Absolute Error), 
RMSE (Root Mean Square Error) and R\textsuperscript{2} score provide a quantitative measure of how efficient a 
model is. Models having lower error and higher accuracy are selected for deployment.

\vspace{0.15cm}

\noindent \textbf{Web-Based Deployment :} Modern employee analytics systems make use of web 
technologies and interactive dashboards to simplify result visualization. Tools like Flask 
framework are used to deploy machine learning models as web applications. This allows 
employees or administrators to interact with the prediction system easily. Through the web 
interface, users can upload data, view results and generate predictions without programming 
knowledge. 


\newpage
\chapter*{
    \centering \vspace{-4cm}
    Chapter - 4\\[0.3cm]
    \textbf{Implementation}
}
\addcontentsline{toc}{chapter}{Chapter 4 – Implementation}

% ------------------ 4.1 CODING ------------------
\section*{4.1 Coding}
\addcontentsline{toc}{section}{4.1 Coding}

\begin{verbatim}
from flask import Flask, render_template, request, jsonify
import os
import pickle
from model_utils import train_models
import pandas as pd
import base64
from io import BytesIO
import matplotlib.pyplot as plt
import seaborn as sns

app = Flask(__name__, template_folder='templates')

# Allow parent folder imports if run from Flask folder
import sys
sys.path.append(os.path.abspath(os.path.join(os.path.dirname(__file__), '..')))

@app.route('/')
def index():
    return render_template('index.html')

@app.route('/app')
def app_page():
    return render_template('app.html')

@app.route('/upload', methods=['POST'])
def upload():
    file = request.files['file']
    os.makedirs('uploads', exist_ok=True)
    filepath = os.path.join('uploads', file.filename)
    file.save(filepath)

    # Train models
    results, features = train_models(filepath)
    df_metrics = pd.DataFrame(results).T

    # --- Bar Chart ---
    fig, ax = plt.subplots(figsize=(8, 5))
    df_metrics.plot(kind='bar', ax=ax)
    plt.title('Model Comparison (MAE, RMSE, R²)')
    plt.xticks(rotation=20)
    plt.tight_layout()

    buf = BytesIO()
    plt.savefig(buf, format='png')
    buf.seek(0)
    bar_chart_base64 = base64.b64encode(buf.read()).decode('utf-8')
    plt.close(fig)

    # --- Heatmap ---
    fig, ax = plt.subplots(figsize=(5, 4))
    corr = df_metrics.corr()
    sns.heatmap(corr, annot=True, cmap='coolwarm', ax=ax)
    plt.title('Metrics Correlation Heatmap')
    plt.tight_layout()

    buf = BytesIO()
    plt.savefig(buf, format='png')
    buf.seek(0)
    heatmap_base64 = base64.b64encode(buf.read()).decode('utf-8')
    plt.close(fig)

    return jsonify({
        "results": results,
        "features": features,
        "bar_chart": bar_chart_base64,
        "heatmap": heatmap_base64
    })

@app.route('/predict', methods=['POST'])
def predict():
    data = request.json
    model_name = data['model']
    features = data['features']

    model_path = os.path.join("IBM Files",
                              f"{model_name.replace(' ', '_')}.pkl")
    scaler_path = os.path.join("IBM Files", "scaler.pkl")

    with open(model_path, "rb") as f:
        model = pickle.load(f)

    with open(scaler_path, "rb") as f:
        scaler = pickle.load(f)

    df = pd.DataFrame([features])
    scaled = scaler.transform(df)
    pred = model.predict(scaled)[0]

    return jsonify({"prediction": float(pred)})

if __name__ == '__main__':
    app.run(debug=False)
\end{verbatim}


% ------------------ FLOWCHART SECTION ------------------
\section*{4.2 Flowchart}
\addcontentsline{toc}{section}{4.2 Flowchart}

\justifying
The flowchart shown below represents the working process of the 
\textbf{Employee Productivity Predictor System}. It illustrates the sequential steps involved 
in uploading a dataset, training machine learning models, generating evaluation metrics, 
visualizing results, and predicting employee productivity. This helps in understanding 
the overall logical workflow of the system.

\vspace{0.8cm}

\begin{center}
    \includegraphics[width=0.28\textwidth]{Flow.png} % Flowchart image
\end{center}

\clearpage



% =========================================
% CHAPTER 5 : INTERFACE
% =========================================

\chapter*{\centering \vspace{-4cm} \textbf{Chapter - 5}\\[4pt] \textbf{Interface}}
\addcontentsline{toc}{chapter}{Interface}

\vspace{0.5cm}

% ---------------- FIGURE 5.1 ----------------
\begin{center}
\includegraphics[width=1\textwidth]{1.png}
\vspace{0.3cm}

{\textbf{Fig : 5.1 User Interface}}
\end{center}

\vspace{1.2cm}

% ---------------- FIGURE 5.2 ----------------
\begin{center}
\includegraphics[width=1\textwidth]{2.png}
\vspace{0.3cm}

{\textbf{Fig : 5.2 CSV File Uploader}}
\end{center}

\vspace{1.2cm}

% ---------------- FIGURE 5.3 ----------------
\begin{center}
\includegraphics[width=0.9\textwidth]{3.png}
\vspace{0.3cm}

{\textbf{Fig : 5.3 Training Result and Visual Comparison}}
\end{center}

\vspace{1.2cm}

% ---------------- FIGURE 5.4 ----------------
\begin{center}
\includegraphics[width=0.90\textwidth]{4.png}
\vspace{0.3cm}

{\textbf{Fig : 5.4 Prediction Result}}
\end{center}

\clearpage

% =========================================
% CHAPTER 6 : FUTURE SCOPE
% =========================================

\chapter*{\centering \vspace{-4cm} \textbf{Chapter - 6}\\[4pt] \textbf{Future Scope}}
\addcontentsline{toc}{chapter}{Future Scope}

\justifying

The prediction of employee productivity using machine learning still has a wide scope for 
improvement and research. With the increase in availability of employee-related data and 
growth of artificial intelligence techniques, the system can be further enhanced to perform 
more accurate and real-time productivity estimation. In future, advanced machine learning 
models can be integrated to handle larger datasets, dynamic behavior and changing workforce 
conditions.

The system can also be extended to monitor employee performance continuously instead of 
depending only on past datasets. Deep learning algorithms and neural networks may be 
incorporated to understand complex behavioral patterns and non-linear relations which 
traditional regression models may not capture effectively. Artificial intelligence techniques 
such as reinforcement learning and hybrid modeling can also be explored to optimize 
employee allocation and scheduling automatically.

Another major area of future scope is the integration of real-time data collection from 
attendance systems, biometric devices, project management tools and HR databases. This 
would enable organizations to measure live productivity values and take immediate 
corrective actions whenever performance goes down. Additionally, explainable AI techniques 
may be applied to provide clear reasoning behind the predictions, so that HR managers can 
understand which factors contribute most to employee productivity.

Moreover, the developed system can be converted into a full-fledged software platform with 
role-based login, customized dashboards, mobile accessibility and cloud deployment. The 
platform may also support industry-specific predictive modules and generate automatic 
suggestions for performance improvement, staff planning and workforce development 
policies.

Thus, future developments can make the system more intelligent, transparent, efficient and 
suitable for real industry implementation.

\clearpage


% =========================================
% REFERENCES
% =========================================

\chapter*{\centering \vspace{-4cm} \textbf{References}}
\addcontentsline{toc}{chapter}{References}

\justifying

\begin{enumerate}

\item Mishra, P. \& Gupta, R. (2021). Predicting Employee Performance using Machine 
Learning Techniques. \textit{International Journal of Scientific Research in Computer 
Science}, Vol. 9.

\item Bishop, C. M. (2006). \textit{Pattern Recognition and Machine Learning}. Springer, e-Book 
Edition.

\item “Machine Learning.” Wikipedia, The Free Encyclopedia. Available at:  
https://en.wikipedia.org/wiki/Machine\_learning

\item Kaggle. Employee Performance and Productivity Dataset (Open Resource). Available 
at: https://www.kaggle.com

\item IBM Developer Documentation. Machine Learning Deployment using Flask 
Framework. Available at: https://developer.ibm.com

\item Scikit-Learn Official Documentation. Available at: https://scikit-learn.org

\item TowardsDataScience. Analytics Techniques for Workforce Productivity. Available 
online at: https://towardsdatascience.com

\end{enumerate}

\clearpage

% =========================================
% APPENDIX
% =========================================
\chapter*{\centering \vspace{-4cm} \textbf{Appendix}}
\addcontentsline{toc}{chapter}{Appendix}

\begin{figure}[H]
\centering
\vspace{4cm}
\includegraphics[width=1\textwidth]{certificateUT.png}
\end{figure}

\end{document}

